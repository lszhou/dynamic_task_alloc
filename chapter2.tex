
%--------------------CHAPTER 2-----------------------------
\chapter{Model of Computation and Definitions}
In this chapter, we will describe our model of computation and give the definitions, which are based on Herlihy
and Wing's \cite{Herlihy:1990:LCC:78969.78972} and Golab, Hadzilacos and Woelfel's \cite{InProc-GHHW2007a}.

The computational model we consider is the standard asynchronous shared memory model with a set of $n$ processes,
denoted as $\{p_1, p_2,...,p_n\}$ , where up to $n-1$ processes may fail by crashing.


\textbf{Type}.
A $type$ $\tau$ is defined as follows \cite{InProc-GHHW2007a},
$$\tau = (\mathcal{S}, s_{init},\mathcal{O},\mathcal{R} ,\delta )$$

where $\mathcal{S}$ is a set of states, $s_{init} \in \mathcal{S}$ is the initial state, $\mathcal{O}$ is a set of
operation types, $\mathcal{R}$ is the set of responses, and
$\delta :\mathcal{S} \times \mathcal{O} \to \mathcal{S} \times \mathcal{R}$ is a state transition mapping.

(\emph{read more explaination about ``type" here in the paper and add some more contents.})




\textbf{Object}.
An object is a sequential implementation (\emph{define what is `implementation'?}) of a type.
(\emph{Have more description based on type definition here.})

An object supports only read and write operations is called a \emph{read/write register} (or just \emph{register}).

A compare-and-swap (CAS) object $\tau$ supports two operations: \texttt{read()} and \texttt{CAS(x,y)}.
Operation \texttt{read()} returns the current state of $\tau$ and leaves the state unchanged. Operation
\texttt{CAS(x,y)} changes the state if and only if the current state of $\tau$ is equal to the given state $x$, i.e,
if current state of $\tau$ equals $x$, then operation \texttt{CAS(x,y)} succeeds, and the state of $\tau$ is changed
to be $y$ and $true$ is returned. Otherwise, operation \texttt{CAS(x,y)} fails, the current state remains unchanged and
$false$ is returned.

In our thesis, we consider the system that supports atomic (\emph{define}) CAS object and atomic read/write register
(\emph{define read and write register}). Processes interact with objects by applying operations on them.



\textbf{History}
(\emph{add more explaination for history!}).
A \emph{history} $H$, obtained by processes executing
operations on shared objects, is a sequence of method \emph{invocation} and \emph{response} events
(\emph{what are invocation and response events based on type definition.}). $H|obj$ of history $H$ is the subsequence of all
invocation and response events in $H$ whose object names are $obj$. If all invocation and response
events in a history $H$ have the same object name $obj$, then the $H|obj = H$. Thus, in the following discussion,
when we discuss the concurrency behavior of a specific objet $obj$, the history $H$ and $H|obj$ are the same.

We use $inv_H(op)$ to denote the invocation of operation $op$ in history $H$ and use $rsp_H(op)$ to
denote the matching response of $op$ in $H$. An invocation event $inv_H(op)$, it is not necessary to have a
matching response. In this case, we say the operation $op$ is \emph{pending}. Otherwise, we call operation \emph{op}
is \emph{complete}.

A history $H$ is \emph{complete} if all operations in $H$ are \emph{complete}. A history $H'$ is an extension of a history $H$
if $H$ is the prefix of $H'$. A history $H'$ is called the completion of an incomplete history $H$ if $H'$, containing the same
operations as $H$ is an extension of $H$ and all operations in $H'$ are complete.

Let $H$ be a complete history, we could associate a time interval $I_H(op) = [inv(op), rsp(op)]$ with each
operation $op$ in $H$. Similarly, for an uncomplete history, we denote the interval with respect to the pending
operation $op$ by $I_H(op) = [inv(op), \infty]$.

A history is \emph{sequential} if the first event is an invocation, and each invocation, except possibly the last
one, is immediately followed by a matching response (\emph{define: matching response?}).



\textbf{Linearization.}
A history $H$ linearizes to a sequential history $S$, if and only if $S$ satisfies the
following conditions: (1) $S$ and the completion of $H$ have the same operations, (2) sequential history $S$ is
valid, and (3) there is a mapping from each time interval $I_H(op)$ to a time point $t_H(op) \in I_H(op)$, such
that the sequential history $S$ could be obtained by sorting the operations in $H$ by their $t_H(op)$ values.

A history is linearizable if and only if there exists a sequential history $S$ that linearizes $H$. In this case,
$S$ is called the linearization of $H$. Each linearization of $H$ defines a point $t_H(op)$. For each operation $op$
in history $H$, we call the point $t_H(op)$ linearization point of $op$. A shared object is linearizable if every
history $H$ of that object is linearizable.



\textbf{Randomness}.
A process can execute local coin flip operation that returns an integer value distributed
uniformly at random from an arbitrary finite set of integers. In the following discussion, we use method
\texttt{random(s)} to return a value which is distributed uniformly at random from set $\{0, 1, 2,..., s-1\}$.



\textbf{Adversary}.
We analyze our algorithm under the assumption of a strong adaptive adversary. At any
point of time, it can see the entire past history and know the states of all processes.

%========================== Update log =================
% Aug.27
% -change some items like from "value" to "state"
% -compeletion of a imcomplete history definition.
