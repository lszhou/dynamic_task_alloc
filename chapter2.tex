
%--------------------CHAPTER 2-----------------------------
\chapter{Model of Computation and Definitions}
In this chapter, we will describe our model of computation and give the definitions, which are based on Herlihy
and Wing's \cite{Herlihy:1990:LCC:78969.78972} and Golab, Hadzilacos and Woelfel's \cite{InProc-GHHW2007a}.

The computational model we consider is the standard asynchronous shared memory model with a set of $n$ processes,
denoted as $\{p_1, p_2,...,p_n\}$ , where up to $n-1$ processes may fail by crashing.


\textbf{Type and Object}.
A $type$ $\tau$ could be defined as an automaton as follows \cite{InProc-GHHW2007a},
$$\tau = (\mathcal{S}, s_{init},\mathcal{O},\mathcal{R} ,\delta )$$

where $\mathcal{S}$ is a set of states, $s_{init} \in \mathcal{S}$ is the initial state, $\mathcal{O}$ is a set of
operation types, $\mathcal{R}$ is the set of responses, and
$\delta :\mathcal{S} \times \mathcal{O} \to \mathcal{S} \times \mathcal{R}$ is a one-to-many state transition mapping.

An $object$ is an implementation of a type. For each type $\tau$, the transition mapping $\delta$ captures the
behaviour of objects of type $\tau$, in the absence of concurreny,
as follows: if a process applies an operation of type $opt$ to an object of type $\tau$ which is in state $s$, the the object
may return to the process a response $r$ and change its states to $s'$ if and only if $(s', r) \in \delta(s, opt)$.

An object supports only \texttt{read()} and \texttt{write(x)} operations is called a $read/write register$
(or just $register$). Operation \texttt{read()} returns the current state of register and leaves the state unchanged.
Operation \texttt{write(x)} changes the state of register to be $x$.

An object supports \texttt{read()} and \texttt{CAS(x,y)} operations is called $compare-and-swap$ (CAS) object.
Operation \texttt{CAS(x,y)} changes the state of the obejct if and only if the current state is equal to the given state $x$, i.e,
if current state equals $x$, then operation \texttt{CAS(x,y)} succeeds, and the state is changed
to be $y$ and $true$ is returned. Otherwise, operation \texttt{CAS(x,y)} fails, the current state remains unchanged and
$false$ is returned.

\textbf{History}
(\emph{add more explaination for history!}).
A \emph{history} $H$, obtained by processes executing
operations on shared objects, is a sequence of method \emph{invocation} and \emph{response} events
(\emph{what are invocation and response events based on type definition.}). $H|obj$ of history $H$ is the subsequence of all
invocation and response events in $H$ whose object names are $obj$. If all invocation and response
events in a history $H$ have the same object name $obj$, then the $H|obj = H$. Thus, in the following discussion,
when we discuss the concurrency behavior of a specific objet $obj$, the history $H$ and $H|obj$ are the same.

We use $inv_H(op)$ to denote the invocation of operation $op$ in history $H$ and use $rsp_H(op)$ to
denote the matching response of $op$ in $H$. An invocation event $inv_H(op)$, it is not necessarily to have a
matching response. In this case, we say the operation $op$ is \emph{pending}. Otherwise, we call operation \emph{op}
is \emph{complete}.

A history $H$ is \emph{complete} if all operations in $H$ are \emph{complete}. A history $H'$ is an extension of a history $H$
if $H$ is the prefix of $H'$. A history $H'$ is called the completion of an incomplete history $H$ if $H'$, containing the same
events as $H$, is an extension of $H$ and each operation in $H'$ is complete.

We say an invocation or response event happens at time $T$ if it is the $T$th event in the history.
Let $H$ be a complete history, we could associate a time interval $I_H(op) = [T_{inv(op)}, T_{rsp(op)}]$ with each
operation $op$ in $H$, where $T_{inv(op)}$ and $T_{rsp(op)}$ is the time when event $inv(op)$ and $rsp(op)$ happens
respectively. Similarly, for an uncomplete history, we denote the interval with respect to the pending
operation $op$ by $I_H(op) = [T_{inv(op)}, \infty]$.

A history is \emph{sequential} if the first event is an invocation, and each invocation, except possibly the last
one, is immediately followed by a matching response (\emph{define: matching response?}).



\textbf{Linearization.}
A history $H$ linearizes to a sequential history $S$, if and only if $S$ satisfies the
following conditions: (1) $S$ and any completion of $H$ have the same operations, (2) sequential history $S$ is
valid, and (3) there is a mapping from each time interval $I_H(op)$ to a time point $t_H(op) \in I_H(op)$, such
that the sequential history $S$ is obtained by sorting the operations in $H$ by their $t_H(op)$ values.

A history is linearizable if and only if $H$ linearizes to some sequential history $S$. In this case,
$S$ is called the linearization of $H$. For each operation $op$ in history $H$, we call the point $t_H(op)$
the linearization point of $op$. A shared object is linearizable if every history $H$ on that object is linearizable.



\textbf{Randomness}.
A process can execute local coin flip operation that returns an integer value distributed
uniformly at random from an arbitrary finite set of integers. In the following discussion, we use method
\texttt{random(s)} to return a value which is distributed uniformly at random from set $\{0, 1, 2,..., s-1\}$.



\textbf{Adversary}.
We analyze our algorithm under the assumption of a strong adaptive adversary. At any
point of time, it can see the entire past history and know the states of all processes.

(\emph{What a process do to execute a program? What is an adversary? What is the functionality of an adversary? (to schedule processes )?
How it schedules processes?})


%========================== Update log =================
% Aug.27
% -change some items like from "value" to "state"
% -compeletion of a imcomplete history definition.
% -some typos
